\chapter{Introduction}\label{ch:intro}

Falls are the second leading cause of accidental injury deaths worldwide, where each year, an estimated 646,000 individuals die from falls globally [1]. This has led to the need for fall detection-based systems to provide an immediate response; however, current methods in place all come with drawbacks. 

Phone-based and smartwatch-based fall detection have risen in popularity. However, in research conducted by \textit{Luque et al.} [2], it was shown that the rate of battery consumption while running complex algorithms was a major drawback. These methods also place responsibility on the individual by making sure they always have their devices on them at all times, and that these devices are charged.

Camera-based fall detection has been another popular choice for fall detection, especially with the prominence of convolutional neural networks (CNN). Unfortunately these methods are supervised learning techniques that require a large amount of data. Real fall data is often unacquariable, as shown in research by \textit{Khan et al.} [3]; falls are rare events, and simulated falls don't encapsulate all aspects of a natural fall.

Privacy concerns also arise when it comes to camera-based fall detection. Data required for camera-based fall detection tend to be a sequence of frames that are stored in video form. This effectively puts individuals under constant surveillance where some third party is storing their activity data. [5] It was shown that one of the most significant drawbacks of camera-based recognition is the central public opinion that this technology causes a threat to individual privacy.

Currently, the biggest issue in fall detection technology using Machine Learning has been the collection of usable data. Research performed by \textit{Schwickert et al.} [4] has shown that 94\% of data gathered for supervised learning techniques and validating models for semi-supervised learning techniques has been done through the use of simulated falls. Amongst this, according to research by \textit{Khan et al.} [3], there is still no standard for fall detection that has been shown to perform consistently well.

Due to these issues, this thesis aims to develop a deep learning-based model to detect human activities and identify falls as anomalies using mmWave sensors while maintaining personal privacy. The model developed during this thesis will also run in real-time to detect falls and alert an individual upon detection. If this model performs at a high specificity in real-time with a low count of false alarms, then this will be taken to facilities in VitalCare for use.  

Chapter 2 sets out to explain in further detail; the motivation behind the topic choice, previous work conducted relating to fall detection, the reasoning behind mmWave sensors, and describe the problem statement. 
Chapter 3 sets out to describe functional requirements, implementation of systems, and technology choice.
Chapter 4 sets out to present the results of this research, and evaluate the results in great detail. Finally, Chapter 5 sets out to conclude this report as well as offer suggestions for future work continuing from this project. 

